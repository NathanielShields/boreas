\section{Additional Air Conditioning Startups}

Attached are lightly edited summaries of a couple of other AC companies I identified that may be of interest.

\subsection{Blue Frontier}

\subsubsection{Overview}
Blue Frontier is a climatetech company developing an air conditioning system that integrates ultra-efficient cooling, humidity control, and energy storage. Using a liquid desiccant-based approach, its technology reduces both energy consumption and the climate impact of traditional HVAC systems. Founded in 2017 and based in Boca Raton, Florida, the company is initially targeting the commercial rooftop HVAC market, which represents a \$20 billion+ addressable market in the U.S.

\subsubsection{Team}
\begin{itemize}
    \item Daniel Betts (Founder \& CEO) – Entrepreneur and engineer with a background in thermal energy systems and renewable technologies.
    \item Matt Graham (COO) – Previously at Tesla Energy, where he helped scale commercial energy solutions.
    \item Steve Lamb (CTO) – Expert in HVAC technology and desiccant-based cooling systems.
\end{itemize}

\subsubsection{Funding and Investors}
Blue Frontier has raised \$20 million in Series A funding (2022), led by:
\begin{itemize}
    \item Breakthrough Energy Ventures
    \item 2150 Urban Tech Sustainability Fund
    \item VoLo Earth Ventures
\end{itemize}

\subsubsection{Value Proposition and Differentiation}
Blue Frontier's system combines dew-point cooling, desiccant dehumidification, and built-in energy storage. This approach offers several advantages:

\begin{itemize}
    \item Energy savings of 50–90\% compared to conventional HVAC units.
    \item Built-in energy storage, reducing peak electricity demand by up to 90\% and enabling greater use of renewable energy.
    \item Reduction of high-GWP refrigerants by 85\%, mitigating HVAC-related climate impact.
    \item Retrofit-ready 5- to 20-ton rooftop HVAC units for commercial real estate.
    \item HVAC-as-a-Service model, offering a subscription-based adoption pathway.
\end{itemize}

\subsubsection{Market and Scalability}
\begin{itemize}
    \item The global commercial HVAC market exceeds \$100 billion.
    \item Initial focus is on commercial buildings in hot and humid climates, where energy savings and dehumidification deliver the greatest return on investment.
    \item Future expansion into residential HVAC, data centers, and industrial cooling.
\end{itemize}

\subsubsection{Competitive Landscape and Risks}
Blue Frontier differentiates itself by integrating air conditioning with energy storage, unlike competitors such as Transaera (MOF-based cooling) and SkyCool Systems (radiative cooling).

Challenges and risks include:
\begin{itemize}
    \item Scaling and manufacturing requirements in a hardware-intensive industry.
    \item Adoption barriers due to long commercial HVAC replacement cycles (15–20 years).
    \item Compliance with ASHRAE standards, utility incentives, and building codes.
\end{itemize}

\subsubsection{Investment Case}
\begin{itemize}
    \item Differentiated technology with a high barrier to entry.
    \item Significant energy efficiency and peak demand reduction.
    \item Strong backing from climatetech investors.
    \item Demand for sustainable cooling in commercial real estate.
\end{itemize}

\subsubsection{Final Verdict}
Blue Frontier has a clear market fit, a scalable business model, and strong investor support.

\subsection{Mojave Energy Systems}

\subsubsection{Overview}
Mojave Energy Systems, founded in 2022 in Sunnyvale, California, focuses on high-efficiency air conditioning with its ArctiDry system, a dedicated outdoor air system (DOAS) that leverages liquid desiccant technology. The company targets the commercial HVAC market, emphasizing energy-efficient dehumidification.

\subsubsection{Leadership Team}
\begin{itemize}
    \item Philip Farese (Founder \& CEO) – Experienced in energy systems and HVAC innovation.
\end{itemize}

\subsubsection{Funding and Investors}
\begin{itemize}
    \item Seed Round (September 2023): \$12.5 million, co-led by At One Ventures and Fifth Wall, with participation from Xerox Ventures.
    \item Series A (December 2024): \$9.5 million, led by Fifth Wall and At One Ventures, with participation from Earth Venture Capital.
\end{itemize}

\subsubsection{Value Proposition}
Mojave’s ArctiDry system offers:
\begin{itemize}
    \item Up to 50\% energy reduction using a high-concentration salt solution for dehumidification.
    \item 20\% lower refrigerant usage, reducing greenhouse gas emissions.
    \item Easy integration with existing infrastructure.
\end{itemize}

\subsubsection{Market Traction and Future Plans}
Since launching ArctiDry in January 2024, Mojave has:
\begin{itemize}
    \item Begun manufacturing and shipping units from its facility in Anderson, South Carolina.
    \item Built a sales partner network with 19 firms, aiming for full U.S. coverage in 2025.
\end{itemize}

\subsubsection{Investment Case}
\begin{itemize}
    \item Focuses on commercial HVAC rather than residential cooling.
    \item Achieves significant energy and refrigerant savings.
    \item Backed by reputable climatetech investors.
    \item Scalable through a partnered distribution network.
\end{itemize}

\subsubsection{Final Verdict}
Mojave Energy Systems presents a strong investment opportunity in climatetech, improving HVAC energy efficiency in commercial buildings. Its differentiated liquid desiccant approach, strong funding, and early market traction position it well for growth.
